
%% bare_jrnl.tex
%% V1.3
%% 2007/01/11
%% by Michael Shell
%% see http://www.michaelshell.org/
%% for current contact information.
%%
%% This is a skeleton file demonstrating the use of IEEEtran.cls
%% (requires IEEEtran.cls version 1.7 or later) with an IEEE journal paper.
%%
%% Support sites:
%% http://www.michaelshell.org/tex/ieeetran/
%% http://www.ctan.org/tex-archive/macros/latex/contrib/IEEEtran/
%% and
%% http://www.ieee.org/



% *** Authors should verify (and, if needed, correct) their LaTeX system  ***
% *** with the testflow diagnostic prior to trusting their LaTeX platform ***
% *** with production work. IEEE's font choices can trigger bugs that do  ***
% *** not appear when using other class files.                            ***
% The testflow support page is at:
% http://www.michaelshell.org/tex/testflow/


%%*************************************************************************
%% Legal Notice:
%% This code is offered as-is without any warranty either expressed or
%% implied; without even the implied warranty of MERCHANTABILITY or
%% FITNESS FOR A PARTICULAR PURPOSE! 
%% User assumes all risk.
%% In no event shall IEEE or any contributor to this code be liable for
%% any damages or losses, including, but not limited to, incidental,
%% consequential, or any other damages, resulting from the use or misuse
%% of any information contained here.
%%
%% All comments are the opinions of their respective authors and are not
%% necessarily endorsed by the IEEE.
%%
%% This work is distributed under the LaTeX Project Public License (LPPL)
%% ( http://www.latex-project.org/ ) version 1.3, and may be freely used,
%% distributed and modified. A copy of the LPPL, version 1.3, is included
%% in the base LaTeX documentation of all distributions of LaTeX released
%% 2003/12/01 or later.
%% Retain all contribution notices and credits.
%% ** Modified files should be clearly indicated as such, including  **
%% ** renaming them and changing author support contact information. **
%%
%% File list of work: IEEEtran.cls, IEEEtran_HOWTO.pdf, bare_adv.tex,
%%                    bare_conf.tex, bare_jrnl.tex, bare_jrnl_compsoc.tex
%%*************************************************************************

% Note that the a4paper option is mainly intended so that authors in
% countries using A4 can easily print to A4 and see how their papers will
% look in print - the typesetting of the document will not typically be
% affected with changes in paper size (but the bottom and side margins will).
% Use the testflow package mentioned above to verify correct handling of
% both paper sizes by the user's LaTeX system.
%
% Also note that the "draftcls" or "draftclsnofoot", not "draft", option
% should be used if it is desired that the figures are to be displayed in
% draft mode.
%
\documentclass[journal]{IEEEtran}
%
% If IEEEtran.cls has not been installed into the LaTeX system files,
% manually specify the path to it like:
% \documentclass[journal]{../sty/IEEEtran}





% Some very useful LaTeX packages include:
% (uncomment the ones you want to load)


% *** MISC UTILITY PACKAGES ***
%
%\usepackage{ifpdf}
% Heiko Oberdiek's ifpdf.sty is very useful if you need conditional
% compilation based on whether the output is pdf or dvi.
% usage:
% \ifpdf
%   % pdf code
% \else
%   % dvi code
% \fi
% The latest version of ifpdf.sty can be obtained from:
% http://www.ctan.org/tex-archive/macros/latex/contrib/oberdiek/
% Also, note that IEEEtran.cls V1.7 and later provides a builtin
% \ifCLASSINFOpdf conditional that works the same way.
% When switching from latex to pdflatex and vice-versa, the compiler may
% have to be run twice to clear warning/error messages.






% *** CITATION PACKAGES ***
%
%\usepackage{cite}
% cite.sty was written by Donald Arseneau
% V1.6 and later of IEEEtran pre-defines the format of the cite.sty package
% \cite{} output to follow that of IEEE. Loading the cite package will
% result in citation numbers being automatically sorted and properly
% "compressed/ranged". e.g., [1], [9], [2], [7], [5], [6] without using
% cite.sty will become [1], [2], [5]--[7], [9] using cite.sty. cite.sty's
% \cite will automatically add leading space, if needed. Use cite.sty's
% noadjust option (cite.sty V3.8 and later) if you want to turn this off.
% cite.sty is already installed on most LaTeX systems. Be sure and use
% version 4.0 (2003-05-27) and later if using hyperref.sty. cite.sty does
% not currently provide for hyperlinked citations.
% The latest version can be obtained at:
% http://www.ctan.org/tex-archive/macros/latex/contrib/cite/
% The documentation is contained in the cite.sty file itself.





% *** GRAPHICS RELATED PACKAGES ***
%
\ifCLASSINFOpdf
  % \usepackage[pdftex]{graphicx}
  % declare the path(s) where your graphic files are
  % \graphicspath{{../pdf/}{../jpeg/}}
  % and their extensions so you won't have to specify these with
  % every instance of \includegraphics
  % \DeclareGraphicsExtensions{.pdf,.jpeg,.png}
\else
  % or other class option (dvipsone, dvipdf, if not using dvips). graphicx
  % will default to the driver specified in the system graphics.cfg if no
  % driver is specified.
  % \usepackage[dvips]{graphicx}
  % declare the path(s) where your graphic files are
  % \graphicspath{{../eps/}}
  % and their extensions so you won't have to specify these with
  % every instance of \includegraphics
  % \DeclareGraphicsExtensions{.eps}
\fi
% graphicx was written by David Carlisle and Sebastian Rahtz. It is
% required if you want graphics, photos, etc. graphicx.sty is already
% installed on most LaTeX systems. The latest version and documentation can
% be obtained at: 
% http://www.ctan.org/tex-archive/macros/latex/required/graphics/
% Another good source of documentation is "Using Imported Graphics in
% LaTeX2e" by Keith Reckdahl which can be found as epslatex.ps or
% epslatex.pdf at: http://www.ctan.org/tex-archive/info/
%
% latex, and pdflatex in dvi mode, support graphics in encapsulated
% postscript (.eps) format. pdflatex in pdf mode supports graphics
% in .pdf, .jpeg, .png and .mps (metapost) formats. Users should ensure
% that all non-photo figures use a vector format (.eps, .pdf, .mps) and
% not a bitmapped formats (.jpeg, .png). IEEE frowns on bitmapped formats
% which can result in "jaggedy"/blurry rendering of lines and letters as
% well as large increases in file sizes.
%
% You can find documentation about the pdfTeX application at:
% http://www.tug.org/applications/pdftex





% *** MATH PACKAGES ***
%
%\usepackage[cmex10]{amsmath}
% A popular package from the American Mathematical Society that provides
% many useful and powerful commands for dealing with mathematics. If using
% it, be sure to load this package with the cmex10 option to ensure that
% only type 1 fonts will utilized at all point sizes. Without this option,
% it is possible that some math symbols, particularly those within
% footnotes, will be rendered in bitmap form which will result in a
% document that can not be IEEE Xplore compliant!
%
% Also, note that the amsmath package sets \interdisplaylinepenalty to 10000
% thus preventing page breaks from occurring within multiline equations. Use:
%\interdisplaylinepenalty=2500
% after loading amsmath to restore such page breaks as IEEEtran.cls normally
% does. amsmath.sty is already installed on most LaTeX systems. The latest
% version and documentation can be obtained at:
% http://www.ctan.org/tex-archive/macros/latex/required/amslatex/math/





% *** SPECIALIZED LIST PACKAGES ***
%
%\usepackage{algorithmic}
% algorithmic.sty was written by Peter Williams and Rogerio Brito.
% This package provides an algorithmic environment fo describing algorithms.
% You can use the algorithmic environment in-text or within a figure
% environment to provide for a floating algorithm. Do NOT use the algorithm
% floating environment provided by algorithm.sty (by the same authors) or
% algorithm2e.sty (by Christophe Fiorio) as IEEE does not use dedicated
% algorithm float types and packages that provide these will not provide
% correct IEEE style captions. The latest version and documentation of
% algorithmic.sty can be obtained at:
% http://www.ctan.org/tex-archive/macros/latex/contrib/algorithms/
% There is also a support site at:
% http://algorithms.berlios.de/index.html
% Also of interest may be the (relatively newer and more customizable)
% algorithmicx.sty package by Szasz Janos:
% http://www.ctan.org/tex-archive/macros/latex/contrib/algorithmicx/




% *** ALIGNMENT PACKAGES ***
%
%\usepackage{array}
% Frank Mittelbach's and David Carlisle's array.sty patches and improves
% the standard LaTeX2e array and tabular environments to provide better
% appearance and additional user controls. As the default LaTeX2e table
% generation code is lacking to the point of almost being broken with
% respect to the quality of the end results, all users are strongly
% advised to use an enhanced (at the very least that provided by array.sty)
% set of table tools. array.sty is already installed on most systems. The
% latest version and documentation can be obtained at:
% http://www.ctan.org/tex-archive/macros/latex/required/tools/


%\usepackage{mdwmath}
%\usepackage{mdwtab}
% Also highly recommended is Mark Wooding's extremely powerful MDW tools,
% especially mdwmath.sty and mdwtab.sty which are used to format equations
% and tables, respectively. The MDWtools set is already installed on most
% LaTeX systems. The lastest version and documentation is available at:
% http://www.ctan.org/tex-archive/macros/latex/contrib/mdwtools/


% IEEEtran contains the IEEEeqnarray family of commands that can be used to
% generate multiline equations as well as matrices, tables, etc., of high
% quality.


%\usepackage{eqparbox}
% Also of notable interest is Scott Pakin's eqparbox package for creating
% (automatically sized) equal width boxes - aka "natural width parboxes".
% Available at:
% http://www.ctan.org/tex-archive/macros/latex/contrib/eqparbox/





% *** SUBFIGURE PACKAGES ***
%\usepackage[tight,footnotesize]{subfigure}
% subfigure.sty was written by Steven Douglas Cochran. This package makes it
% easy to put subfigures in your figures. e.g., "Figure 1a and 1b". For IEEE
% work, it is a good idea to load it with the tight package option to reduce
% the amount of white space around the subfigures. subfigure.sty is already
% installed on most LaTeX systems. The latest version and documentation can
% be obtained at:
% http://www.ctan.org/tex-archive/obsolete/macros/latex/contrib/subfigure/
% subfigure.sty has been superceeded by subfig.sty.



%\usepackage[caption=false]{caption}
%\usepackage[font=footnotesize]{subfig}
% subfig.sty, also written by Steven Douglas Cochran, is the modern
% replacement for subfigure.sty. However, subfig.sty requires and
% automatically loads Axel Sommerfeldt's caption.sty which will override
% IEEEtran.cls handling of captions and this will result in nonIEEE style
% figure/table captions. To prevent this problem, be sure and preload
% caption.sty with its "caption=false" package option. This is will preserve
% IEEEtran.cls handing of captions. Version 1.3 (2005/06/28) and later 
% (recommended due to many improvements over 1.2) of subfig.sty supports
% the caption=false option directly:
%\usepackage[caption=false,font=footnotesize]{subfig}
%
% The latest version and documentation can be obtained at:
% http://www.ctan.org/tex-archive/macros/latex/contrib/subfig/
% The latest version and documentation of caption.sty can be obtained at:
% http://www.ctan.org/tex-archive/macros/latex/contrib/caption/




% *** FLOAT PACKAGES ***
%
%\usepackage{fixltx2e}
% fixltx2e, the successor to the earlier fix2col.sty, was written by
% Frank Mittelbach and David Carlisle. This package corrects a few problems
% in the LaTeX2e kernel, the most notable of which is that in current
% LaTeX2e releases, the ordering of single and double column floats is not
% guaranteed to be preserved. Thus, an unpatched LaTeX2e can allow a
% single column figure to be placed prior to an earlier double column
% figure. The latest version and documentation can be found at:
% http://www.ctan.org/tex-archive/macros/latex/base/



%\usepackage{stfloats}
% stfloats.sty was written by Sigitas Tolusis. This package gives LaTeX2e
% the ability to do double column floats at the bottom of the page as well
% as the top. (e.g., "\begin{figure*}[!b]" is not normally possible in
% LaTeX2e). It also provides a command:
%\fnbelowfloat
% to enable the placement of footnotes below bottom floats (the standard
% LaTeX2e kernel puts them above bottom floats). This is an invasive package
% which rewrites many portions of the LaTeX2e float routines. It may not work
% with other packages that modify the LaTeX2e float routines. The latest
% version and documentation can be obtained at:
% http://www.ctan.org/tex-archive/macros/latex/contrib/sttools/
% Documentation is contained in the stfloats.sty comments as well as in the
% presfull.pdf file. Do not use the stfloats baselinefloat ability as IEEE
% does not allow \baselineskip to stretch. Authors submitting work to the
% IEEE should note that IEEE rarely uses double column equations and
% that authors should try to avoid such use. Do not be tempted to use the
% cuted.sty or midfloat.sty packages (also by Sigitas Tolusis) as IEEE does
% not format its papers in such ways.


%\ifCLASSOPTIONcaptionsoff
%  \usepackage[nomarkers]{endfloat}
% \let\MYoriglatexcaption\caption
% \renewcommand{\caption}[2][\relax]{\MYoriglatexcaption[#2]{#2}}
%\fi
% endfloat.sty was written by James Darrell McCauley and Jeff Goldberg.
% This package may be useful when used in conjunction with IEEEtran.cls'
% captionsoff option. Some IEEE journals/societies require that submissions
% have lists of figures/tables at the end of the paper and that
% figures/tables without any captions are placed on a page by themselves at
% the end of the document. If needed, the draftcls IEEEtran class option or
% \CLASSINPUTbaselinestretch interface can be used to increase the line
% spacing as well. Be sure and use the nomarkers option of endfloat to
% prevent endfloat from "marking" where the figures would have been placed
% in the text. The two hack lines of code above are a slight modification of
% that suggested by in the endfloat docs (section 8.3.1) to ensure that
% the full captions always appear in the list of figures/tables - even if
% the user used the short optional argument of \caption[]{}.
% IEEE papers do not typically make use of \caption[]'s optional argument,
% so this should not be an issue. A similar trick can be used to disable
% captions of packages such as subfig.sty that lack options to turn off
% the subcaptions:
% For subfig.sty:
% \let\MYorigsubfloat\subfloat
% \renewcommand{\subfloat}[2][\relax]{\MYorigsubfloat[]{#2}}
% For subfigure.sty:
% \let\MYorigsubfigure\subfigure
% \renewcommand{\subfigure}[2][\relax]{\MYorigsubfigure[]{#2}}
% However, the above trick will not work if both optional arguments of
% the \subfloat/subfig command are used. Furthermore, there needs to be a
% description of each subfigure *somewhere* and endfloat does not add
% subfigure captions to its list of figures. Thus, the best approach is to
% avoid the use of subfigure captions (many IEEE journals avoid them anyway)
% and instead reference/explain all the subfigures within the main caption.
% The latest version of endfloat.sty and its documentation can obtained at:
% http://www.ctan.org/tex-archive/macros/latex/contrib/endfloat/
%
% The IEEEtran \ifCLASSOPTIONcaptionsoff conditional can also be used
% later in the document, say, to conditionally put the References on a 
% page by themselves.





% *** PDF, URL AND HYPERLINK PACKAGES ***
%
\usepackage{url}
% url.sty was written by Donald Arseneau. It provides better support for
% handling and breaking URLs. url.sty is already installed on most LaTeX
% systems. The latest version can be obtained at:
% http://www.ctan.org/tex-archive/macros/latex/contrib/misc/
% Read the url.sty source comments for usage information. Basically,
% \url{my_url_here}.

% *** CUSTOM ***
\usepackage{bibentry} % inline bibentries 
\makeatletter 
\renewcommand\BR@b@bibitem[2][]{\BR@bibitem[#1]{#2}\BR@c@bibitem{#2}} 
\makeatother
\nobibliography* %reuse biblography config for inline bibentries 

\usepackage{todonotes} %todo notes                                                           
%\usepackage[disable]{todonotes} % to disable/un-display all todonotes at once uncomment this

% *** Do not adjust lengths that control margins, column widths, etc. ***
% *** Do not use packages that alter fonts (such as pslatex).         ***
% There should be no need to do such things with IEEEtran.cls V1.6 and later.
% (Unless specifically asked to do so by the journal or conference you plan
% to submit to, of course. )


% correct bad hyphenation here
\hyphenation{op-tical net-works semi-conduc-tor}


\begin{document}
%
% paper title
% can use linebreaks \\ within to get better formatting as desired
\title{Test \LaTeX~File for Blockchain Bib}
%
%
% author names and IEEE memberships
% note positions of commas and nonbreaking spaces ( ~ ) LaTeX will not break
% a structure at a ~ so this keeps an author's name from being broken across
% two lines.
% use \thanks{} to gain access to the first footnote area
% a separate \thanks must be used for each paragraph as LaTeX2e's \thanks
% was not built to handle multiple paragraphs
%

%\author{Michael~Shell,~\IEEEmembership{Member,~IEEE,}
%        John~Doe,~\IEEEmembership{Fellow,~OSA,}
%        and~Jane~Doe,~\IEEEmembership{Life~Fellow,~IEEE}% <-this % stops a space
%\thanks{M. Shell is with the Department
%of Electrical and Computer Engineering, Georgia Institute of Technology, Atlanta,
%GA, 30332 USA e-mail: (see http://www.michaelshell.org/contact.html).}% <-this % stops a space
%\thanks{J. Doe and J. Doe are with Anonymous University.}% <-this % stops a space
%\thanks{Manuscript received April 19, 2005; revised January 11, 2007.}}

% note the % following the last \IEEEmembership and also \thanks - 
% these prevent an unwanted space from occurring between the last author name
% and the end of the author line. i.e., if you had this:
% 
% \author{....lastname \thanks{...} \thanks{...} }
%                     ^------------^------------^----Do not want these spaces!
%
% a space would be appended to the last name and could cause every name on that
% line to be shifted left slightly. This is one of those "LaTeX things". For
% instance, "\textbf{A} \textbf{B}" will typeset as "A B" not "AB". To get
% "AB" then you have to do: "\textbf{A}\textbf{B}"
% \thanks is no different in this regard, so shield the last } of each \thanks
% that ends a line with a % and do not let a space in before the next \thanks.
% Spaces after \IEEEmembership other than the last one are OK (and needed) as
% you are supposed to have spaces between the names. For what it is worth,
% this is a minor point as most people would not even notice if the said evil
% space somehow managed to creep in.



% The paper headers
%\markboth{Journal of \LaTeX\ Class Files,~Vol.~6, No.~1, January~2007}%
%{Shell \MakeLowercase{\textit{et al.}}: Bare Demo of IEEEtran.cls for Journals}
% The only time the second header will appear is for the odd numbered pages
% after the title page when using the twoside option.
% 
% *** Note that you probably will NOT want to include the author's ***
% *** name in the headers of peer review papers.                   ***
% You can use \ifCLASSOPTIONpeerreview for conditional compilation here if
% you desire.




% If you want to put a publisher's ID mark on the page you can do it like
% this:
%\IEEEpubid{0000--0000/00\$00.00~\copyright~2007 IEEE}
% Remember, if you use this you must call \IEEEpubidadjcol in the second
% column for its text to clear the IEEEpubid mark.



% use for special paper notices
%\IEEEspecialpapernotice{(Invited Paper)}




% make the title area
\maketitle


\begin{abstract}
%\boldmath
This file includes all references from the blockchain.bib plus some
abstracts of various publications.
\end{abstract}
% IEEEtran.cls defaults to using nonbold math in the Abstract.
% This preserves the distinction between vectors and scalars. However,
% if the journal you are submitting to favors bold math in the abstract,
% then you can use LaTeX's standard command \boldmath at the very start
% of the abstract to achieve this. Many IEEE journals frown on math
% in the abstract anyway.

% Note that keywords are not normally used for peerreview papers.
\begin{IEEEkeywords}
Test, IEEEtran, journal, \LaTeX, paper, template, blockchain, bibliography.
\end{IEEEkeywords}






% For peer review papers, you can put extra information on the cover
% page as needed:
% \ifCLASSOPTIONpeerreview
% \begin{center} \bfseries EDICS Category: 3-BBND \end{center}
% \fi
%
% For peerreview papers, this IEEEtran command inserts a page break and
% creates the second title. It will be ignored for other modes.
\IEEEpeerreviewmaketitle



%\section{Introduction}
% The very first letter is a 2 line initial drop letter followed
% by the rest of the first word in caps.
% 
% form to use if the first word consists of a single letter:
% \IEEEPARstart{A}{demo} file is ....
% 
% form to use if you need the single drop letter followed by
% normal text (unknown if ever used by IEEE):
% \IEEEPARstart{A}{}demo file is ....
% 
% Some journals put the first two words in caps:
% \IEEEPARstart{T}{his demo} file is ....
% 
% Here we have the typical use of a "T" for an initial drop letter
% and "HIS" in caps to complete the first word.
%\IEEEPARstart{T}{his} demo file is intended to serve as a ``starter file''
%for IEEE journal papers produced under \LaTeX\ using
%IEEEtran.cls version 1.7 and later.
% You must have at least 2 lines in the paragraph with the drop letter
% (should never be an issue)
%I wish you the best of success.

%\hfill mds
 
%\hfill January 11, 2007

% needed in second column of first page if using \IEEEpubid
%\IEEEpubidadjcol

%%% Include papers %%% 
\section{ZombieCoin: powering next-generation botnets with bitcoin}
\bibentry{ali2015zombiecoin}

\textbf{Abstract:} 
Botnets are the preeminent source of online crime and arguably the greatest threat to the Internet infrastructure.
In this paper, we present ZombieCoin, a botnet command-and-control (C\&C) mechanism that runs on the Bitcoin network. 
ZombieCoin offers considerable advantages over existing C\&C techniques, most notably the fact that Bitcoin is designed to resist the very regulatory processes currently used to combat botnets. We believe this is a desirable avenue botmasters may explore in the near future and our work is intended as a first step towards devising effective countermeasures.

\section{PoW-Based Distributed Cryptography with No Trusted Setup}
\bibentry{andrychowicz2015pow}

\textbf{Abstract:} 
Motivated by the recent success of Bitcoin we study the question of constructing distributed cryptographic protocols in a fully peer-to-peer scenario (without any trusted setup) under the assumption that the adversary has limited computing power. We propose a formal model for this scenario and then we construct the following protocols working in it: (i) a broadcast protocol secure under the assumption that the honest parties have computing power that is some non-negligible fraction of computing power of the adversary (this fraction can be small, in particular it can be much less than 1/2), (ii) a protocol for identifying a set of parties such that the majority of them is honest, and every honest party belongs to this set (this protocol works under the assumption that the majority of computing power is controlled by the honest parties). Our broadcast protocol can be used to generate an unpredictable beacon (that can later serve, e.g., as a genesis block for a new cryptocurrency). The protocol from Point (ii) can be used to construct arbitrary multiparty computation protocols. Our main tool for checking the computing power of the parties are the Proofs of Work (Dwork and Naor, CRYPTO 92). Our broadcast protocol is built on top of the classical protocol of Dolev and Strong (SIAM J. on Comp. 1983). Although our motivation is mostly theoretic, we believe that our ideas can lead to practical implementations (probably after some optimizations and simplifications). We discuss some possible applications of our protocols at the end of the paper. We stress however that the goal of this paper is not to propose new cryptocurrencies or to analyze the existing ones.

\section{Babaioff, Moshe and Dobzinski, Shahar and Oren, Sigal and Zohar, Aviv }
\bibentry{babaioff2012red}

\textbf{Abstract:} 
We study scenarios in which the goal is to ensure that some information will propagate through a large network of nodes. In these scenarios all nodes that are aware of the information compete for the same prize, and thus have an incentive not to propagate information. One example for such a scenario is the 2009 DARPA Network Challenge (finding red balloons). We give special attention to a second domain, Bitcoin, a decentralized electronic currency system. Bitcoin, which has been getting a large amount of public attention over the last year, represents a radical new approach to monetary systems which has appeared in policy discussions and in the popular press [3, 11].  Its cryptographic fundamentals have largely held up even as its usage has become increasingly widespread.  We find, however, that it exhibits a fundamental problem of a different nature, based on how its incentives are structured.  We propose a modification to the protocol that can fix this problem. Bitcoin relies on a peer-to-peer network to track transactions that are performed with the currency. For this purpose, every transaction a node learns about should betransmitted to its neighbors in the network. As the protocol is currently defined and implemented, it does not provide an incentive for nodes to broadcast transactions they are aware of. In fact, it provides a strong incentive not to do so. Our solution is to augment the protocol with a scheme that rewards information propagation. We show that our proposed scheme succeeds in setting the correct incentives, that it is Sybil-proof, and that it requires only a small payment overhead, all this is achieve with iterated elimination of dominated strategies. We provide lower bounds on the overhead that is required to implement schemes with the stronger solution concept of Dominant Strategies, indicating that such schemes might be impractical.

\section{Hashcash-a denial of service counter-measure}
\bibentry{back2002hashcash}

\textbf{Abstract:} Hashcash was originally proposed as a mechanism to throttle systematic abuse of un-metered internet resources such as email, and anonymous remailers in May 1997. Five years on, this paper captures in one place the various applications, improvements suggested and related subsequent publications, and describes initial experience from experiments using hashcash. The hashcash CPU cost-function computes a token which can be used as a proof-of-work. Interactive and noninteractive variants of cost-functions can be constructed which can be used in situations where the server can issue a challenge (connection oriented interactive protocol), and where it can not (where the communication is store–and–forward, or packet oriented) respectively.

\section{Bitter to better - how to make bitcoin a better currency}
\bibentry{barber2012bitterbetter}

\textbf{Abstract:} 
Bitcoin is a distributed digital currency which has attracted a substantial number of users. We perform an in-depth investigation to understand what made Bitcoin so successful, while decades of research on cryptographic e-cash has not lead to a large-scale deployment. We ask also how Bitcoin could become a good candidate for a long-lived stable currency. In doing so, we identify several issues and attacks of Bitcoin, and propose suitable techniques to address them.

\section{Cryptocurrencies without proof of work}
\bibentry{bentov2014cryptocurrencies}

\textbf{Abstract:} 
We study decentralized cryptocurrency protocols in which the participants do not deplete physical scarce resources. Such protocols commonly rely on Proof of Stake, i.e., on mechanisms that extend voting power to the stakeholders of the system. We offer analysis of existing protocols that have a substantial amount of popularity. We then present our novel pure Proof of Stake protocols, and argue that they help in mitigating problems that the existing protocols exhibit.

\section{Bitcoin over Tor isn't a good idea}
\bibentry{biryukov2015bitcoin}

\textbf{Abstract:} 
Bitcoin is a decentralized P2P digital currency in which coins are generated by a distributed set of miners and transaction are broadcasted via a peer-to-peer network. While Bitcoin provides some level of anonymity (or rather pseudonymity) by encouraging the users to have any number of random-looking Bitcoin addresses, recent research shows that this level of anonymity is rather low. This encourages users to connect to the Bitcoin network through anonymizers like Tor and motivates development of default Tor functionality for popular mobile SPV clients. In this paper we show that combining Tor and Bitcoin creates an attack vector for the deterministic and stealthy man-in-the-middle attacks. A low-resource attacker can gain full control of information flows between all users who chose to use Bitcoin over Tor. In particular the attacker can link together user’s transactions regardless of pseudonyms used, control which Bitcoin blocks and transactions are relayed to the user and can delay or discard user’s transactions and blocks. In collusion with a powerful miner double-spending attacks become possible and a totally virtual Bitcoin reality can be created  for such set of users. Moreover, we show how an attacker can fingerprint users and then recognize them and learn their IP address when they decide to connect to the Bitcoin network directly.

\section{On Bitcoin as a public randomness source}
\bibentry{bonneau2015random}

\textbf{Abstract:} 
We formalize the use of Bitcoin as a source of publiclyverifiable randomness. As a side-effect of Bitcoin’s proof-of-work-based consensus system random values are broadcast every time new blocks are mined. We can derive strong lower bounds on the computational min-entropy in each block: currently at least 68 bits of min-entropy are produced every 10 minutes from which one can derive over 32 nearuniform bits using standard extractor techniques. We show that any attack on this beacon would form an attack on Bitcoin itself and hence have a monetary cost that we can bound unlike any other construction for a public randomness beacon in the literature. In our simplest construction we show that a lottery producing a single unbiased bit is manipulation-resistant against an attacker with a stake of less than 50 bitcoins in the output or about US\$12 000 today. Finally we propose making the beacon output available to smart contracts and demonstrate that this simple tool enables a number of interesting applications.

\section{Sok: Research Perspectives and Challenges for Bitcoin and Cryptocurrencies}
\bibentry{bonneau2015research}

\textbf{Abstract:} 
Bitcoin has emerged as the most successful cryptographic currency in history. Within two years of its quiet launch in 2009, Bitcoin grew to comprise billions of dollars of economic value despite only cursory analysis of the system’s design. Since then a growing literature has identified hidden-but-important properties of the system, discovered attacks, proposed promising alternatives, and singled out difficult future challenges. Meanwhile a large and vibrant open-source community has proposed and deployed numerous modifications and extensions. We provide the first systematic exposition Bitcoin and the many related cryptocurrencies or ‘altcoins.’ Drawing from a scattered body of knowledge, we identify three key components of Bitcoin’s design that can be decoupled. This enables a more insightful analysis of Bitcoin’s properties and future stability. We map the design space for numerous proposed modifications, providing comparative analyses for alternative consensus mechanisms, currency allocation mechanisms, computational puzzles, and key management tools. We survey anonymity issues in Bitcoin and provide an evaluation framework for analyzing a variety of privacy-enhancing proposals. Finally we provide new insights on what we term disintermediation protocols, which absolve the need for trusted intermediaries in an interesting set of applications. We identify three general disintermediation strategies and provide a detailed comparison.

\section{EthIKS: Using Ethereum to audit a CONIKS key transparency log}
\bibentry{bonneau2016ethiks}

\textbf{Abstract:} 
CONIKS is a proposed key transparency system which enables a centralized service provider to maintain an auditable yet privacypreserving directory of users' public keys. In  the original CONIKS design, users must monitor that their data is correctly included in every published snapshot of the directory, necessitating either slow updates or trust in an unspecied third-party to audit that the data structure has stayed consistent.  We  demonstrate  that  the  data  structures  for CONIKS  are very similar to those used in Ethereum, a consensus computation platform  with  a  Turing-complete  programming  environment.  We  can  take advantage  of  this  to  embed  the  core  CONIKS  data  structures  into  an Ethereum contract with only minor modications. Users may then trust the Ethereum network to audit the data structure for consistency and non-equivocation. Users who do not trust (or are unaware of) Ethereum can  self-audit  the  CONIKS  data  structure  as  before. We  have  implemented a prototype contract for our hybrid EthIKS scheme, demonstrating that it adds only modest bandwidth overhead to CONIKS proofs and costs hundredths of pennies per key update in fees at today's rates.

\section{On Decentralizing Prediction Markets and Order Books}
\bibentry{clark2014decentralizing}

\textbf{Abstract:} 
We propose techniques for decentralizing prediction markets and order books, utilizing
Bitcoin’s security model and consensus mechanism. Decentralization of prediction markets offers several
key advantages over a centralized market: no single entity governs over the market, all transactions are
transparent in the block chain, and anybody can participate pseudonymously to either open a new
market or place bets in an existing one. We provide trust agility: each market has its own specified
arbiter and users can choose to interact in markets that rely on the arbiters they trust. We also provide
a transparent, decentralized order book that enables order execution on the block chain in the presence
of potentially malicious miners.

\section{On Scaling Decentralized Blockchains}
\bibentry{croman2016scaling}

\textbf{Abstract:} 
The increasing popularity of blockchain-based cryptocurrencies has made scalability a primary and urgent concern. We analyze how fundamental  and  circumstantial  bottlenecks  in  Bitcoin  limit  the  ability of its current peer-to-peer overlay network to support substantially higher throughputs and lower latencies. Our results suggest that reparameterization of block size and intervals should be viewed only as a first increment toward achieving next-generation, high-load blockchain protocols, and major advances will additionally require a basic rethinking of technical  approaches.  We  offer  a  structured  perspective  on  the  design space for such approaches. Within this perspective, we enumerate and briefly  discuss  a  number  of  recently  proposed  protocol  ideas  and  offer several new ideas and open challenges.

\section{Provisions: Privacy-preserving proofs of solvency for Bitcoin exchanges }
\bibentry{dagher2015provisions}

\textbf{Abstract:} 
Bitcoin exchanges function like banks, securely holding their customers’bitcoins on their behalf. Several exchanges have suffered catastrophic losses with customers permanently losing their savings. A proof of solvency demonstrates that the exchange controls sufficient reserves to settle each customer’s account. We introduce Provisions, a privacy-preserving proof of solvency whereby an exchange does not have to disclose its Bitcoin addresses; total holdings or liabilities; or any information about its customers. We also propose an extension which prevents exchanges from colluding to cover for each other’s losses. We have implemented Provisions and it offers practical computation times and proof sizes even for a large Bitcoin exchange with millions of customers.

\section{Centrally Banked Cryptocurrencies}
\bibentry{danezis2015centrally}

\textbf{Abstract:} 
Current cryptocurrencies, starting with Bitcoin, build a decentralized blockchain-based transaction ledger, maintained through proofs-of-work that also serve to generate a monetary supply. Such decentralization has benefits, such as independence from national political control, but also significant limitations in terms of computational costs and scalability. We introduce RSCoin, a cryptocurrency framework in which central banks maintain complete control over the monetary supply, but rely on a distributed set of authorities, or mintettes, to prevent double-spending. While monetary policy is centralized, RSCoin still provides strong transparency and auditability guarantees. We demonstrate, both theoretically and experimentally, the benefits of a modest degree of centralization, such as the elimination of wasteful hashing and a scalable system for avoiding doublespending attacks.

\section{Making Bitcoin Exchanges Transparent}
\bibentry{decker2015making}

\textbf{Abstract:} 
Bitcoin exchanges are a vital component of the Bitcoin ecosystem. They are a gateway from the classical economy to the cryptocurrency economy, facilitating the exchange between fiat currency and bitcoins. However, exchanges are also single points of failure, operating outside the Bitcoin blockchain, requiring users to entrust them with their funds in order to operate. In this work we present a solution, and a proof-of-concept implementation, that allows exchanges to prove their solvency, without publishing any information of strategic importance.

\section{A first look at the usability of bitcoin key management}
\bibentry{eskandari2015usability}

\textbf{Abstract:} 
Bitcoin users are directly or indirectly forced to deal with public key cryptography, which has a number of security and usability challenges that differ from the password-based authentication underlying most online banking services. Users must ensure that keys are simultaneously accessible, resistant to digital theft and resilient to loss. In this paper, we contribute an evaluation framework for comparing Bitcoin key management approaches, and conduct a broad usability evaluation of six representative Bitcoin clients. We find that Bitcoin shares many of the fundamental challenges of key management known from other domains, but that Bitcoin may present a unique opportunity to rethink key management for end users.

\section{The bitcoin backbone protocol: Analysis and applications}
\bibentry{garay2015bitcoinbackbone}

\textbf{Abstract:} 
Bitcoin is the first and most popular decentralized cryptocurrency to date. In this work, we extract and analyze the core of the Bitcoin protocol, which we term the Bitcoin backbone, and prove two of its fundamental properties which we call common prefix and chain quality in the static setting where the number of players remains fixed. Our proofs hinge on appropriate and novel assumptions on the “hashing power” of the adversary relative to network synchronicity; we show our results to be tight under high synchronization. Next, we propose and analyze applications that can be built “on top” of the backbone protocol, specifically focusing on Byzantine agreement (BA) and on the notion of a public transaction ledger. Regarding BA, we observe that Nakamoto’s suggestion falls short of solving it, and present a simple alternative which works assuming that the adversary’s hashing power isbounded by 1/3. The public transaction ledger captures the essence of Bitcoin’s operation as a cryptocurrency, in the sense that it guarantees the liveness and persistence of committed transactions. Based on this notion we describe and analyze the Bitcoin system as well as a more elaborate BA protocol, proving them secure assuming high network synchronicity and that the adversary’s hashing power is strictly less than 1/2, while the adversarial bound needed for security decreases as the network desynchronizes.

\section{Tampering with the Delivery of Blocks and Transactions in Bitcoin}
\bibentry{gervais2015tampering}

\textbf{Abstract:} 
Given the increasing adoption of Bitcoin, the number of transactions and the block sizes within the system are only expected to increase. To sustain its correct operation in spite of its ever-increasing use, Bitcoin implements a number of necessary optimizations and scalability measures. These measures limit the amount of information broadcast in the system to the minimum necessary. In this paper, we show that current scalability measures adopted by Bitcoin come at odds with the security of the system. More specifically, we show that an adversary can exploit these measures in order to effectively delay the propagation of transactions and blocks to specific nodes—without causing a network partitioning in the system. We show that this allows the adversary to easily mount Denial-of-Service attacks, considerably increase its mining advantage in the network, and double-spend transactions in spite of the current countermeasures adopted by Bitcoin. Based on our results, we propose a number of countermeasures in order to enhancethe security of Bitcoin without deteriorating its scalability.

\section{Bitcoin Blockchain Dynamics: the Selfish-Mine Strategy in the Presence of Propagation Delay }
\bibentry{gobel2015simulation}

\textbf{Abstract:} 
In the context of the ‘selfish-mine’ strategy pro- posed by Eyal and Sirer we study the effect of propagation delay on the evolution of the Bitcoin blockchain. First we u se a simplified Markov model that tracks the contrasting states of belief about the blockchain of a small pool of miners and the ‘rest of the community’ to establish that the use of block-hi ding strategies such as selfish-mine causes the rate of product ion of orphan blocks to increase. Then we use a spatial Poisson proc ess model to study values of Eyal and Sirer’s parameter γ which denotes the proportion of the honest community that mine on a previously-secret block released by the pool in response to the mining of a block by the honest community. Finally we use discrete-event simulation to study the behaviour of a netwo rk of Bitcoin miners a proportion of which is colluding in usin g the selfish-mine strategy under the assumption that there i s a propagation delay in the communication of information betw een miners

\section{Securing Bitcoin wallets via a new DSA/ECDSA threshold signature scheme }
\bibentry{goldfeder2015threshold}

\textbf{Abstract:} 
The Bitcoin ecosystem has suffered frequent thefts and losses affecting both businesses and individuals. Due to the irreversibility, automation, and pseudonymity of transactions, Bitcoin currently lacks support for the sophisticated internal control systems deployed by modern businesses to deter fraud. To address this problem, we present the first threshold signature scheme compatible with Bitcoin’s ECDSA signatures and show how distributed Bitcoin wallets can be built using this primitive. For businesses, we show how our distributed wallets can be used to systematically eliminate single points of failure at every stage of the flow of bitcoins through the system. For individuals, we design, implement, and evaluate a two-factor secure Bitcoin wallet.

\section{Hierarchical deterministic Bitcoin wallets that tolerate key leakage (short paper)}
\bibentry{gutoski2015hierarchical}

\textbf{Abstract:} 
A Bitcoin wallet is a set of private keys known to a user and which allow that user to spend any Bitcoin associated with those keys. In a hierarchical deterministic (HD) wallet, child private keys are generated pseudorandomly from a master private key, and the corresponding child public keys can be generated by anyone with knowledge of the master public key. These wallets have several interesting applications including Internet retail, trustless audit, and a treasurer allocating funds among departments. A specification of HD wallets has even been accepted as Bitcoin standard BIP32. Unfortunately, in all existing HD wallets—including BIP32 wallets—an attacker can easily recover the master private key given the master public key and any child private key. This vulnerability precludes use cases such as a combined treasurer-auditor, and some in the Bitcoin community have suspected that this vulnerability cannot be avoided. We propose a new HD wallet that is not subject to this vulnerability. Our HD wallet can tolerate the leakage of up to m private keys with a master public key size of O(m). We prove that breaking our HD wallet is at least as hard as the so-called “one more” discrete logarithm problem.

\section{An empirical study of Namecoin and lessons for decentralized namespace design}
\bibentry{kalodner2015namecoinempirical}

\textbf{Abstract:} 
Secure decentralized namespaces have recently become possible due to cryptocurrency technology. They enable a censorship-resistant domainname system outside the control of any single entity, among other applications. Namecoin, a fork of Bitcoin, is the most prominent example. We initiate the study of decentralized namespaces and the market for names in such systems. Our extensive empirical analysis of Namecoin reveals a system in disrepair. Indeed, our methodology for detecting ''squatted'' and otherwise inactive domains reveals that among Namecoin’s roughly 120,000 registered domain names, a mere 28 are not squatted and have nontrivial content. Further, we develop techniques for detecting transfers of domains in the Namecoin block chain and provide evidence that the market for domains is thin-tononexistent. We argue that the state of the art in mechanism design for decentralized namespace markets is lacking. We propose a model of utility of different names to different participants, and articulate desiderata of a decentralized namespace in terms of this utility function. We use this model to explore the design

\section{Misbehavior in Bitcoin: A Study of Double-Spending and Accountability}
\bibentry{karame2015misbehavior}

\textbf{Abstract:} 
Bitcoin is a decentralized payment system that relies on Proof-of-Work (PoW) to resist double-spending through a distributed timestamping service. To ensure the operation and security of Bitcoin, it is essential that all transactions and their order of execution are available to all Bitcoin users. Unavoidably, in such a setting, the security of transactions comes at odds with transaction privacy. Motivated by the fact that transaction confirmation in Bitcoin requires tens of minutes, we analyze the conditions for performing successful double-spending attacks against fast payments in Bitcoin, where the time between the exchange of currency and goods is short (in the order of a minute). We show that unless new detection techniques are integrated in the Bitcoin implementation, double-spending attacks on fast payments succeed with considerable probability and can be mounted at low cost. We propose a new and lightweight countermeasure that enables the detection of double-spending attacks in fast transactions. In light of such misbehavior, accountability becomes crucial. We show that in the specific case of Bitcoin, accountability complements privacy. To illustrate this tension, we provide accountability and privacy definition for Bitcoin, and we investigate analytically and empirically the privacy and accountability provisions in Bitcoin.

\section{How to Use Bitcoin to Play Decentralized Poker }
\bibentry{kumaresan2015poker}

\textbf{Abstract:} 
Back and Bentov (arXiv 2014) and Andrychowicz et al. (Security and Privacy 2014) introduced techniques to perform secure multiparty computations on Bitcoin. Among other things, these works constructed lottery protocols that ensure that any party that aborts after learning the outcome pays a monetary penalty to all other parties. Following this, Andrychowicz et al. (Bitcoin Workshop 2014) and concurrently Bentov and Kumaresan (Crypto 2014) extended the solution to arbitrary secure function
evaluation while guaranteeing fairness in the following sense: any party that aborts after learning the output pays a monetary penalty to all parties that did not learn the output. Andrychowicz et al. (Bitcoin Workshop 2014) also suggested extending to scenarios where parties receive a payoff according to the output of a secure function evaluation, and outlined a 2-party protocol for the same that in addition satisfies the notion of fairness described above. In this work, we formalize,
generalize, and construct multiparty protocols for the primitive suggested by Andrychowicz et al. We call this primitive secure cash distribution with penalties. Our formulation of secure cash distribution with penalties poses it as a multistage reactive functionality (i.e., more general than secure function evaluation) that provides a way to securely implement smart contracts in a decentralized setting, and consequently suffices to capture a wide variety of stateful computations involving data
and/or money, such as decentralized auctions, markets, and games such as poker, etc. Our protocol realizing secure cash distribution with penalties works in a hybrid model where parties have access to a claim-or-refund transaction functionality $F^{*}_{CR}$ which can be efficiently realized in (a variant of) Bitcoin, and is otherwise independent of the Bitcoin ecosystem. We emphasize that our protocol is dropout-tolerant in the sense that any party that drops out during the protocol is forced to pay a monetary penalty to all other parties. Our formalization and construction generalize both secure computation with penalties of Bentov and Kumaresan (Crypto 2014), and secure lottery with penalties of Andrychowicz et al. (Security and Privacy 2014).


\section{Inclusive block chain protocols }
\bibentry{lewenberg2015inclusive}

\textbf{Abstract:} 
Distributed cryptographic protocols such as Bitcoin and Ethereum use a data structure known as the block chain to synchronize a global log of events between nodes in their network. Blocks, which are batches of updates to the log, reference the parent they are extending, and thus form the structure of a chain. Previous research has shown that the mechanics of the block chain and block propagation are constrained: if blocks are created at a high rate compared to their propagation time in the network, many conflicting blocks are created and performance suffers greatly. As a result of the low block creation rate required to keep the system within safe parameters, transactions take long to securely confirm, and their throughput is greatly limited. We propose an alternative structure to the chain that allows for operation at much higher rates. Our structure consists of a directed acyclic graph of blocks (the block DAG). The DAG structure is created by allowing blocks to reference multiple predecessors, and allows for more “forgiving” transaction acceptance rules that incorporate transactions even from seemingly conflicting blocks. Thus, larger blocks that take longer to propagate can be tolerated by the system, and transaction volumes can be increased. Another deficiency of block chain protocols is that they favor more connected nodes that spread their blocks faster—fewer of their blocks conflict. We show that with our system the advantage of such highly connected miners is greatly reduced. On the negative side, attackers that attempt to maliciously reverse transactions can try to use the forgiving nature of the DAG structure to lower the costs of their attacks. We provide a security analysis of the protocol and show that such attempts can be easily countered.

\section{Two-factor authentication for the Bitcoin protocol}
\bibentry{mann2015two}

\textbf{Abstract:} 
We show how to realize two-factor authentication for a Bitcoin wallet employing the two-party ECDSA signature protocol adapted from MacKenzie \& Reiter (2004). We also present a prototypic implementation of a Bitcoin wallet that offers both: two-factor authentication and verification over a separate channel. Since we use a smart phone as the second authentication factor, our solution can be used with hardware already available to most users and the user experience is quite similar to the existing online banking authentication methods.

\section{Cryptographic Currencies from a Tech-Policy Perspective: Policy Issues and Technical Directions }
\bibentry{mcreynolds2015cryptographic}

\textbf{Abstract:} 
We study legal and policy issues surrounding crypto currencies, such as Bitcoin, and how those issues interact with technical design options. With an interdisciplinary team, we consider in depth a variety of  issues  surrounding  law,  policy,  and  crypto  currencies | such  as  the physical location where a crypto currency's value exists for jurisdictional and other purposes, the regulation of anonymous or pseudonymous currencies, and challenges as virtual currency protocols and laws evolve. We reflect  onhow  different  technical  directions  may  interact  with  the  relevant  laws  and  policies,  raising  key  issues  for  both  policy  experts  and technologists.

\section{Bringing Deployable Key Transparency to End Users }
\bibentry{melara2015keytransparency}

\textbf{Abstract:} 
We present CONIKS, an end-user key verification service capable of integration in end-to-end encrypted communication systems. CONIKS builds on related designs for transparency of web server certificates but solves several new challenges specific to key verification for end users. In comparison to prior designs, CONIKS enables more efficient monitoring and auditing of keys, allowing small organizations to effectively audit even very large key servers. CONIKS users can efficiently monitor their own key bindings for consistency, downloading less than 20 kB per day to do so even for a provider with billions of users. CONIKS users and providers can collectively audit providers for non-equivocation, and this requires downloading a constant 2.5 kB per day regardless of server size. Unlike any previous proposal, CONIKS also preserves the level of privacy offered by today’s major communication services, hiding the list of usernames present and even allowing providers to conceal the total number of users in the system.

\section{Anonymous byzantine consensus from moderately-hard puzzles: A model for bitcoin}
\bibentry{miller2014anonymous}

\textbf{Abstract:} 
We present a formal model of synchronous processes without distinct identifiers (i.e., anonymous processes) that communicate using one-way public broadcasts. Our main contribution is a proof that the Bitcoin protocol achieves consensus in this model, except for a negligible probability, when Byzantine faults make up less than half the network. The protocol is scalable, since the running time and message complexity are all independent of the size of the network, instead depending only on the relative computing power of the faulty processes. We also introduce a requirement that the protocol must tolerate an arbitrary number of passive clients that receive broadcasts but can not send. This leads to a tight 2f + 1 resilience bound. 

\section{Nonoutsourceable Scratch-Off Puzzles to Discourage Bitcoin Mining Coalitions }
\bibentry{miller2015nonoutsourceable}

\textbf{Abstract:} 
An implicit goal of Bitcoin’s reward structure is to diffuse network influence over a diverse, decentralized population of individual participants. Indeed, Bitcoin’s security claims rely on no single entity wielding a sufficiently large portion of the network’s overall computational power. Unfortunately, rather than participating independently, most Bitcoin miners join coalitions called mining pools in which a central pool administrator largely directs the pool’s activity, leading to a consolidation of power. Recently, the largest mining pool has accounted for more than half of network’s total mining capacity. Relatedly, “hosted mining” service providers offer their clients the benefit of economiesof-scale, tempting them away from independent participation. We argue that the prevalence of mining coalitions is due to a limitation of the Bitcoin proof-of-work puzzle – specifically, that it affords an effective mechanism for enforcing cooperation in a coalition. We present several definitions and constructions for “nonoutsourceable” puzzles that thwart such enforcement mechanisms, thereby deterring coalitions. We also provide an implementation and benchmark results for our schemes to show they are practical.

\section{Discovering bitcoin's public topology and influential nodes }
\bibentry{miller2015topology}

\textbf{Abstract:} 
The  Bitcoin  network  relies  on  peer-to-peer  broadcast to distribute pending transactions and confirmed blocks. The topology over which this broadcast is distributed affects which nodes have advantages and whether some attacks are feasible.  As such, it is particularly important to understand not just which nodes participate in the Bitcoin network, but how they are connected. In this paper, we introduce AddressProbe, a technique that  discovers  peer-to-peer  links  in  Bitcoin,  and  apply this to the live topology.  To support AddressProbe and other tools, we develop CoinScope, an infrastructure to manage short, but large-scale experiments in Bitcoin. We analyze  the  measured  topology  to  discover  both  highdegree nodes and a well connected giant component. Yet, efficient propagation over the Bitcoin backbone does not necessarily result in a transaction being accepted into the block chain. We introduce a “decloaking” method to find influential nodes in the topology that are well connected to a mining pool. Our results find that in contrast to Bitcoin’s idealized vision of spreading mining responsibility to each node, mining pools are prevalent and hidden: roughly  2%  of  the  (influential)  nodes  represent  threequarters of the mining power.


\section{Bitcoin: A Peer-to-Peer Electronic Cash System }
\bibentry{nakamoto2008bitcoin}

\textbf{Abstract:} 
A purely peer-to-peer version of electronic cash would allow online payments to be sent directly from one party to another without going through a financial institution. Digital signatures provide part of the solution, but the main benefits are lost if a trusted third party is still required to prevent double-spending. We propose a solution to the double-spending problem using a peer-to-peer network. The network timestamps transactions by hashing them into an ongoing chain of hash-based proof-of-work, forming a record that cannot be changed without redoing the proof-of-work. The longest chain not only serves as proof of the sequence of events witnessed, but proof that it came from the largest pool of CPU power. As long as a majority of CPU power is controlled by nodes that are not cooperating to attack the network, they'll generate the longest chain and outpace attackers. The network itself requires minimal structure. Messages are broadcast on a best effort basis, and nodes can leave and rejoin the network at will, accepting the longest proof-of-work chain as proof of what happened while they were gone.

\section{Micropayments for Decentralized Currencies }
\bibentry{pass2015micropayments}

\textbf{Abstract:} 
Electronic financial transactions in the US, even those enabled by Bitcoin, have relatively high transaction costs. As a result, it becomes infeasible to make micropayments, i.e. payments that are pennies or fractions of a penny. To circumvent the cost of recording all transactions, Wheeler (1996) and Rivest (1997) suggested the notion of a probabilistic payment, that is, one implements payments that have expected value on the order of micro pennies by running an appropriately biased lottery for a larger payment. While there have been quite a few proposed solutions to such lottery-based micropayment schemes, all these solutions rely on a trusted third party to coordinate the transactions; furthermore, to implement these systems in today’s economy would require a a global change to how either banks or electronic payment companies (e.g., Visa and Mastercard) handle transactions. We put forth a new lottery-based micropayment scheme for any ledger-based transaction system, that can be used today without any change to the current infrastructure. We implement our scheme in a sample web application and show how a single server can handle thousands of micropayment requests per second. We analyze how the scheme can work at Internet scale.

\section{Analysis of Hashrate-Based Double Spending }
\bibentry{rosenfeld2014doublespending}

\textbf{Abstract:} 
Bitcoin ([?]) is the world’s first decentralized digital currency. Its main technical innovation is the use of a blockchain and hash-based proof of work to synchronize transactions and prevent double-spending the currency. While the qualitative nature of this system is well understood, there is widespread confusion about its quantitative aspects and how they relate to attack vectors and their countermeasures. In this paper we take a look at the stochastic processes underlying typical attacks and their resulting probabilities of success.

\section{Liar, Liar, Coins on Fire!: Penalizing Equivocation By Loss of Bitcoins }
\bibentry{ruffing2015liar}

\textbf{Abstract:} 
We show that equivocation, i.e., making conflicting statements to others in a distributed protocol, can be monetarily disincentivized by the use of crypto-currencies such as Bitcoin. To this end, we design completely decentralized non-equivocation contracts, which make it possible to penalize an equivocating party by the loss of its money. At the core of these contracts, there is a novel cryptographic primitive called accountable assertions, which reveals the party’s Bitcoin credentials if it equivocates. Non-equivocation contracts are particularly useful for distributed systems that employ public append-only logs to protect data integrity, e.g., in cloud storage and social networks. Moreover, as double-spending in Bitcoin is a special case of equivocation, the contracts enable us to design a payment protocol that allows a payee to receive funds at several unsynchronized points of sale, while being able to penalize a double-spending payer after the fact.

\section{Increasing Anonymity in Bitcoin}
\bibentry{saxena2014increasing}

\textbf{Abstract:} 
Bitcoin prevents double-spending using the blockchain   a public  ledger  kept  with  every  client.  Every  single  transaction  till  date  is present in this ledger. Due to this  true anonymity is not present in bit- coin. We present a method to enhance anonymity in bitcoin-type cryp- tocurrencies.  In  the  blockchain   each  block  holds  a  list  of  transactions linking the sending and receiving addresses. In our modifyed protocol the transactions (and blocks) do not contain any such links. Using this  we obtain a far higher degree of anonymity. Our method uses a new primi- tive known as composite signatures . Our security is based on the hardness of the Computation Die-Hellman assumption in bilinear maps.

\section{Accelerating Bitcoin's Transaction Processing. Fast Money Grows on Trees, Not Chains}
\bibentry{sompolinsky2013accelerating}

\textbf{Abstract:} 
Bitcoin is a potentially disruptive new crypto-currency based on a decentralized open-source protocol which is gradually gaining popularity. Perhaps the most important question that will affect Bitcoin’s success, is whether or not it will be able to scale to support the high volume of transactions required from a global currency system. We investigate the restrictions on the rate of transaction processing in Bitcoin as a function of both the bandwidth available to nodes and the network delay, both of which lower the efficiency of Bitcoin’s transaction processing. The security analysis done by Bitcoin’s creator Satoshi Nakamoto [12] assumes that block propagation delays are negligible compared to the time between blocks—an assumption that does not hold when the protocol is required to process transactions at high rates. We improve upon the original analysis and remove this assumption. Using our results, we are able to give bounds on the number of transactions per second the protocol can handle securely. Building on previously published measurements by Decker and Wattenhofer [5], we show these bounds are currently more restrictive by an order of magnitude than the bandwidth needed to stream all transactions. We additionally show how currently planned improvements to the protocol, namely the use of transaction hashes in blocks (instead of complete transaction records), will dramatically alleviate these restrictions. Finally, we present an easily implementable modification to the way Bitcoin constructs its main data structure, the blockchain, that immensely improves security from attackers, especially when the network operates at high rates. This improvement allows for further increases in the number of transactions processed per second. We show that with our proposed modification, significant speedups can be gained in confirmation time of transactions as well. The block generation rate can be securely increased to more than one block per second – a 600 fold speedup compared to today’s rate, while still allowing the network to processes many transactions per second.

\section{Bitcoin and Beyond: A Technical Survey on Decentralized Digital Currencies}
\bibentry{tschorsch2015bitcoin}

\textbf{Abstract:} 
Besides attracting a billion dollar economy  Bitcoin revolu- tionized the field of digital currencies and influenced many adjacent areas. This also induced significant scientific inter- est. In this survey  we unroll and structure the manyfold results and research directions. We start by introducing the Bitcoin protocol and its building blocks. From there we continue to explore the design space by discussing existing contributions and results. In the process  we deduce the fun- damental structures and insights at the core of the Bitcoin protocol and its applications. As we show and discuss  many key ideas are likewise applicable in various other fields  so that their impact reaches far beyond Bitcoin itself.

\section{From Pretty Good to Great: Enhancing PGP Using Bitcoin and the Blockchain}
\bibentry{wilson2015pretty}

\textbf{Abstract:} 
PGP is built upon a Distributed Web of Trust in which a user’s trustworthiness is established by others who can vouch through a digital signature for that user’s identity. Preventing its wholesale adoption are a number of inherent weaknesses to include (but not limited to) the following: 1) Trust Relationships are built on a subjective honor system, 2) Only first degree relationships can be fully trusted, 3) Levels of trust are difficult to quantify with actual values, and 4) Issues with the Web of Trust itself (Certification and Endorsement). Although the security that PGP provides is proven to be reliable, it has largely failed to garner large scale adoption. In this paper, we propose several novel contributions to address the aforementioned issues with PGP and associated Web of Trust. To address the subjectivity of the Web of Trust, we provide a new certificate format based on Bitcoin which allows a user to verify a PGP certificate using Bitcoin identity-verification transactions - forming first degree trust relationships that are tied to actual values (i.e., number of Bitcoins transferred during transaction). Secondly, we present the design of a novel Distributed PGP key server that leverages the Bitcoin transaction blockchain to store and retrieve Bitcoin-Based PGP certificates. Lastly, we provide a web prototype application that demonstrates several of these capabilities in an actual environment.

\section{Bitcoin: under the hood}
\bibentry{zohar2015bitcoin}

\textbf{Abstract:} 
I JUST WANT to report that I successfully traded 10,000 bitcoins for pizza,” wrote user laszlo on the Bitcoin forums in May 2010—reporting on what has been recognized as the first item in history to be purchased with bitcoins. By the end of 2013, about five years after its initial launch, Bitcoin has exceeded everyone’s expectations as its value rose beyond the \$1,000 mark, making laszlo’s spent bitcoins worth millions of dollars. This meteoric rise in value has fueled many stories in the popular press and has turned a group of early enthusiasts into millionaires. Stories of Bitcoin’s mysterious creator, Satoshi Nakamoto, and of illegal markets hidden in the darknet have added to the hype. But what is Bitcoin’s innovation? Is the buzz surrounding the new cryptocurrency justified, or will it turn out to be a modern tulip mania? To truly evaluate Bitcoin’s novelty, its potential impact, and the challenges it faces, we must look past the hype and delve deeper into the details of the protocol.

\section{Decentralizing Privacy: Using Blockchain to Protect Personal Data }
\bibentry{zyskind2015decentralizing}

\textbf{Abstract:} 
The recent increase in reported incidents of surveillance and security breaches compromising users’ privacy call into question the current model, in which third-parties collect and control massive amounts of personal data. Bitcoin has demonstrated in the financial space that trusted, auditable computing is possibleusing a decentralized network of peers accompanied by a public ledger.  In  this  paper,  we  describe  a  decentralized  personal  data management  system  that  ensures  users  own  and  control  their data.  We  implement  a  protocol  that  turns  a  blockchain  into  an automated access-control manager that does not require trust in a third party. Unlike Bitcoin, transactions in our system are not strictly  financial  –  they  are  used  to  carry  instructions,  such  as storing,  querying  and  sharing  data.  Finally,  we  discuss  possible future  extensions  to  blockchains  that  could  harness  them  into  a well-rounded solution for trusted computing problems in society.



% An example of a floating figure using the graphicx package.
% Note that \label must occur AFTER (or within) \caption.
% For figures, \caption should occur after the \includegraphics.
% Note that IEEEtran v1.7 and later has special internal code that
% is designed to preserve the operation of \label within \caption
% even when the captionsoff option is in effect. However, because
% of issues like this, it may be the safest practice to put all your
% \label just after \caption rather than within \caption{}.
%
% Reminder: the "draftcls" or "draftclsnofoot", not "draft", class
% option should be used if it is desired that the figures are to be
% displayed while in draft mode.
%
%\begin{figure}[!t]
%\centering
%\includegraphics[width=2.5in]{myfigure}
% where an .eps filename suffix will be assumed under latex, 
% and a .pdf suffix will be assumed for pdflatex; or what has been declared
% via \DeclareGraphicsExtensions.
%\caption{Simulation Results}
%\label{fig_sim}
%\end{figure}

% Note that IEEE typically puts floats only at the top, even when this
% results in a large percentage of a column being occupied by floats.


% An example of a double column floating figure using two subfigures.
% (The subfig.sty package must be loaded for this to work.)
% The subfigure \label commands are set within each subfloat command, the
% \label for the overall figure must come after \caption.
% \hfil must be used as a separator to get equal spacing.
% The subfigure.sty package works much the same way, except \subfigure is
% used instead of \subfloat.
%
%\begin{figure*}[!t]
%\centerline{\subfloat[Case I]\includegraphics[width=2.5in]{subfigcase1}%
%\label{fig_first_case}}
%\hfil
%\subfloat[Case II]{\includegraphics[width=2.5in]{subfigcase2}%
%\label{fig_second_case}}}
%\caption{Simulation results}
%\label{fig_sim}
%\end{figure*}
%
% Note that often IEEE papers with subfigures do not employ subfigure
% captions (using the optional argument to \subfloat), but instead will
% reference/describe all of them (a), (b), etc., within the main caption.


% An example of a floating table. Note that, for IEEE style tables, the 
% \caption command should come BEFORE the table. Table text will default to
% \footnotesize as IEEE normally uses this smaller font for tables.
% The \label must come after \caption as always.
%
%\begin{table}[!t]
%% increase table row spacing, adjust to taste
%\renewcommand{\arraystretch}{1.3}
% if using array.sty, it might be a good idea to tweak the value of
% \extrarowheight as needed to properly center the text within the cells
%\caption{An Example of a Table}
%\label{table_example}
%\centering
%% Some packages, such as MDW tools, offer better commands for making tables
%% than the plain LaTeX2e tabular which is used here.
%\begin{tabular}{|c||c|}
%\hline
%One & Two\\
%\hline
%Three & Four\\
%\hline
%\end{tabular}
%\end{table}


% Note that IEEE does not put floats in the very first column - or typically
% anywhere on the first page for that matter. Also, in-text middle ("here")
% positioning is not used. Most IEEE journals use top floats exclusively.
% Note that, LaTeX2e, unlike IEEE journals, places footnotes above bottom
% floats. This can be corrected via the \fnbelowfloat command of the
% stfloats package.



%\section{Conclusion}
%The conclusion goes here.





% if have a single appendix:
%\appendix[Proof of the Zonklar Equations]
% or
%\appendix  % for no appendix heading
% do not use \section anymore after \appendix, only \section*
% is possibly needed

% use appendices with more than one appendix
% then use \section to start each appendix
% you must declare a \section before using any
% \subsection or using \label (\appendices by itself
% starts a section numbered zero.)
%


%\appendices
%\section{Proof of the First Zonklar Equation}
%Appendix one text goes here.

% you can choose not to have a title for an appendix
% if you want by leaving the argument blank
%\section{}
%Appendix two text goes here.


% use section* for acknowledgement
%\section*{Acknowledgment}


%The authors would like to thank...


% Can use something like this to put references on a page
% by themselves when using endfloat and the captionsoff option.
\ifCLASSOPTIONcaptionsoff
  \newpage
\fi



% trigger a \newpage just before the given reference
% number - used to balance the columns on the last page
% adjust value as needed - may need to be readjusted if
% the document is modified later
%\IEEEtriggeratref{8}
% The "triggered" command can be changed if desired:
%\IEEEtriggercmd{\enlargethispage{-5in}}

% references section

% can use a bibliography generated by BibTeX as a .bbl file
% BibTeX documentation can be easily obtained at:
% http://www.ctan.org/tex-archive/biblio/bibtex/contrib/doc/
% The IEEEtran BibTeX style support page is at:
% http://www.michaelshell.org/tex/ieeetran/bibtex/

\bibliographystyle{apalike}
% argument is your BibTeX string definitions and bibliography database(s)
\bibliography{../blockchain}
\nocite{*}

%
% <OR> manually copy in the resultant .bbl file
% set second argument of \begin to the number of references
% (used to reserve space for the reference number labels box)
%\begin{thebibliography}{1}

%\bibitem{IEEEhowto:kopka}
%H.~Kopka and P.~W. Daly, \emph{A Guide to \LaTeX}, 3rd~ed.\hskip 1em plus
%  0.5em minus 0.4em\relax Harlow, England: Addison-Wesley, 1999.
%
%\end{thebibliography}

% biography section
% 
% If you have an EPS/PDF photo (graphicx package needed) extra braces are
% needed around the contents of the optional argument to biography to prevent
% the LaTeX parser from getting confused when it sees the complicated
% \includegraphics command within an optional argument. (You could create
% your own custom macro containing the \includegraphics command to make things
% simpler here.)
%\begin{biography}[{\includegraphics[width=1in,height=1.25in,clip,keepaspectratio]{mshell}}]{Michael Shell}
% or if you just want to reserve a space for a photo:

%\begin{IEEEbiography}{Michael Shell}
%Biography text here.
%\end{IEEEbiography}

% if you will not have a photo at all:
%\begin{IEEEbiographynophoto}{John Doe}
%Biography text here.
%\end{IEEEbiographynophoto}

% insert where needed to balance the two columns on the last page with
% biographies
%\newpage

%\begin{IEEEbiographynophoto}{Jane Doe}
%Biography text here.
%\end{IEEEbiographynophoto}

% You can push biographies down or up by placing
% a \vfill before or after them. The appropriate
% use of \vfill depends on what kind of text is
% on the last page and whether or not the columns
% are being equalized.

%\vfill

% Can be used to pull up biographies so that the bottom of the last one
% is flush with the other column.
%\enlargethispage{-5in}



% that's all folks
\end{document}


