\section{Discovering bitcoin's public topology and influential nodes }
\bibentry{miller2015topology}

\textbf{Abstract:} 
The  Bitcoin  network  relies  on  peer-to-peer  broadcast to distribute pending transactions and confirmed blocks. The topology over which this broadcast is distributed affects which nodes have advantages and whether some attacks are feasible.  As such, it is particularly important to understand not just which nodes participate in the Bitcoin network, but how they are connected. In this paper, we introduce AddressProbe, a technique that  discovers  peer-to-peer  links  in  Bitcoin,  and  apply this to the live topology.  To support AddressProbe and other tools, we develop CoinScope, an infrastructure to manage short, but large-scale experiments in Bitcoin. We analyze  the  measured  topology  to  discover  both  highdegree nodes and a well connected giant component. Yet, efficient propagation over the Bitcoin backbone does not necessarily result in a transaction being accepted into the block chain. We introduce a “decloaking” method to find influential nodes in the topology that are well connected to a mining pool. Our results find that in contrast to Bitcoin’s idealized vision of spreading mining responsibility to each node, mining pools are prevalent and hidden: roughly  2%  of  the  (influential)  nodes  represent  threequarters of the mining power.

